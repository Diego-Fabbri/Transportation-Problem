\documentclass[a4paper,12pt,titlepage]{article}
\usepackage[utf8]{inputenc} 
\usepackage{tikz,pgf}
\usepackage{indentfirst}
\usepackage{amsfonts}
\usepackage[english]{babel}
%Url e Bookmarks of output PDF 
\usepackage{hyperref}
\hypersetup{
	colorlinks=true,
	linkcolor=blue,
	filecolor=magenta,      
	urlcolor=cyan,
%	pdftitle={Document title},
	bookmarks=true,
	%pdfpagemode=FullScreen,
}
\usepackage{rotating}
\usepackage{tabularx}
\usepackage{multirow} 
\usepackage{lscape}
\usepackage{tikz}
%to insert PDF files
\usepackage[final]{pdfpages}
%--Packages--
\usepackage{eurosym}
\usepackage{graphicx} \usepackage{verbatim}
\usepackage{graphics}
\usepackage{tikz,pgf}
\usepackage{indentfirst}
\usepackage{amsfonts}
\usepackage{graphicx}
\usepackage{amsmath}
\usepackage{amsmath,amssymb,amsthm,textcomp}
\usepackage{enumerate}
\usepackage{multicol}
\usepackage{tikz}
\usepackage{geometry}
\usepackage{mathtools}
\usepackage{amsmath}
\usepackage{verbatim}
\usepackage{amsmath,amssymb,mathrsfs}
\usepackage{xcolor}
\usepackage{graphicx,color,listings}
\frenchspacing 
\usepackage{geometry}
\usepackage{rotating}
\usepackage{caption}
\usepackage{xcolor}
\usepackage{listings}
%Cool maths printing
\usepackage{amsmath}
%PseudoCode
\usepackage{algorithm2e}


\begin{document}
\section*{Transportation Problem}
Goods are produced at $m$ different supply centers $i=1,2,...,m$.\\
The supply produced at supply center $i$ is $S_i$, $i=1,2,...,m$.\\
The demand for the good comes from $n$ different demand centers, $d_j$ $j=1,2,...,n$.\\
Cost of shipping one unit from supply center $i$ to demand center $j$ is $c_{ij}$.\\
Variables $x_{ij}$ define the number of units shipped from supply center $i$ to demand center $j$.
\subsection*{Feasibility condition}
\begin{equation*}
\sum_{j=1}^{n} d_j \leq \sum_{i=1}^{m} S_i
\end{equation*}
\subsection*{Math Formulation}
\begin{equation}
\sum_{i=1}^{m}\sum_{j=1}^{n} c_{ij} \cdot x_{ij}
\tag{1}
\end{equation}
\begin{equation}
\sum_{j=1}^{n} x_{ij} \leq S_i \qquad i=1,2,...,m
\tag{2}
\end{equation}
\begin{equation}
\sum_{i=1}^{m}x_{ij} = d_j \qquad j=1,2,...,n
\tag{3}
\end{equation}
\begin{equation}
x_{ij} \geq 0 \qquad i=1,2,...,m,\,\, j=1,2,...,n
\tag{4}
\end{equation}
In this problem you can also assume that variables $x_{ij}$ take on integer
values (and non-negative ones), it depends on the good you are dealing with.
\end{document}